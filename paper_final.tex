%%
%% This is file `example/ch_concln.tex',
%% generated with the docstrip utility.
%%
%% The original source files were:
%%
%% install/buptgraduatethesis.dtx  (with options: `ch-concln')
%% 
%% This file is a part of the example of BUPTGraduateThesis.
%% 

\chapter{总结与展望}
\section{工作总结}

本课题以面向司法领域的多标签分类为主要研究内容,充分调研现有的相关算法模型,理解现有研究工作中的不足,从标签不均衡问题和标签关联性问题出发,研究并实现了两种司法多标签分类模型:基于成对注意力机制的模型和基于递归注意力机制的模型。

首先,本课题提出了一种基于动态成对注意力机制的模型,通过引入法条定义之间的关联关系,构造注意力机制矩阵,该模型可以通过这种方式用高频标签影响低频标签缓解标签不均衡问题。之后通过一个动态阈值机制,针对每个标签,根据不同条件,生成阈值决定预测标签的边界概率。最后,通过采用多任务学习的方式,在进行多标签分类任务的同时联合学习阈值预测器,使得模型具有更好的泛化性能。通过数据集上的实验,证明了这种模型的有效性。

其次,本课题提出了一种基于递归注意力机制的模型,该模型首先通过自注意力机制,获取案情描述和法条独立自身的重要信息,之后经过递归单元模拟法官在审判案件过程中,来回交替阅读案情描述与法条定义的过程,通过这种方式,学习案情描述与法条定义之间复杂的交互信息,得到更深层次的语义特征,使得模型不仅能考虑标签之间的关联信息,还能考虑案情与法条之间的交互信息。通过在多个大型数据集上的实验验证,证明了该模型的有效性。

最后,本课题设计实现了一个原型系统,对算法模型在实际场景下的应用进行了功能测试。

\section{工作展望}
在DPAM模型中,本课题使用TextCNN用于获得案情表达。然而,在司法领域,案情中的一些关键词,包括谋杀,抢劫,同样对于法官区分不同的案情有所帮助。本课题对裁判文书上这些关键信息的抽取进行了一部分实验,在未来的工作中,本课题将分析司法多标签分类中关键词与案情描述之间的关系。此外,目前的研究多数集中于文本本身的信息,在司法领域,由很多外部知识以非结构化的图形式体现,本课题将探索通过图的方法引入外部知识,分析其对判决预测问题性能的影响。


% \newpage

% \chapterbib
