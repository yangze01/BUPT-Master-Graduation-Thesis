%%
%% This is file `example/ch_intro.tex',
%% generated with the docstrip utility.
%%
%% The original source files were:
%%
%% install/buptgraduatethesis.dtx  (with options: `ch-intro')
%% 
%% This file is a part of the example of BUPTGraduateThesis.
%% 

\chapter{绪论}
\section{研究背景及意义}
随着互联网的普及和发展,各行各业都积累了大量的数据,为了应对信息爆炸带来的挑战,迫切需要一些自动化的手段帮助人们解决现实生活中的一些实际问题。如何通过计算机手段,将实际生活中人们的需求转换为计算机可解决的问题,成为我们需要越发关注的问题。

传统的机器学习分类任务中,通常一个实例只包含一种标签,不同类别之间是互斥的,例如:人脸识别中,一个人的头像只能对应一种类别;新闻分类中,一篇新闻文本只对应于一种类型的新闻。而现实生活中的很多问题,往往一个实例对应多种标签,即多标签分类问题,例如:一个歌曲可以被划分为多个不同的流派;一张图片可能包含多种场景对象;生物信息学中,一个未标记的蛋白质含有多重功能;一部电影可能即属于战争又属于历史。然而,在多标签分类研究中经常出现一些问题,比如不同的标签出现的次数可能会相差很多,标签之间存在相互之间的关联,分别称为标签不均衡问题和标签关联性问题。解决这些问题对多标签分类更好地应用于实际生活有重要意义。

在司法领域,根据事实描述判定犯罪者触发了哪些法律是一项非常繁琐的工作,法官通常需要参考一些相关案例来确定具体涉及的法律条文,这项工作是十分耗时并且需要专业知识的。司法判决预测可以根据事实描述进行判决结果预测,这项技术对司法辅助系统是十分有用的,一方面,它可以提供低成本,高质量的司法咨询服务;另一方面,它可以作为法官和律师等专业人士的参考。 因此,研究如何自动化地解决判决预测问题是十分重要的。



\section{国内外研究现状}
\subsection{司法判决预测}
作为司法智能化的传统任务,自动判决预测已经被研究了数十年。在早期的研究中,研究者通常将判决文书通过统计分析进行建模\cite{LiuC03, kort1957predicting, nagel1963applying, ulmer1963quantitative, keown1980mathematical},这些方法聚焦在如何用数学的方法对判决文书进行案例分析,而没有将判决预测考虑在内。
现有的大多数现有的工作将这个任务当作文本分类问题进行处理,研究者通常通过从文本中抽取有效的特征并利用机器学习方法进行判决预测。Liu等人通过从历史数据中分抽取重要信息,采用KNN方法用于决定司法案例的诉讼理由\cite{LiuCH04}。Latz等人提出了一个基于树的模型用于预测美国最高法院法官的判决行为。Sulea等人采用案例描述,规则以及案例时间作为特征信息,开发了一个采用多个SVM模型结果进行集成预测的系统\cite{Sulea2017Exploring}。Carvalho等人针对这个任务提出了一种两步方法,第一步,通过混合n-gram模型根据给定事实描述召回若干个法条,接下来,采用机器学习方法决定这些法条中哪些是真正相关的\cite{carvalho2015lexical}。这些方法需要花费大量的精力进行特征设计,并且很难进行大规模应用。

受到神经网络在各领域成功的启发\cite{Kim14, BordesGWB12, LuongSLVZ15},研究者开始将神经网络技术引入到判决预测任务中。Luo等人提出一种基于注意力机制的神经网络方法在一个统一的框架下联合建模判决预测任务和相关法条抽取任务\cite{luo2017learning}。Zhong等人将任务之间的依赖关系建模成一个有向图,并将这种拓扑结构引入判决预测任务中\cite{ZhongGTX0S18}。Hu等人提出

\subsection{多标签分类}


\section{研究内容}
\subsection{基于动态阈值的Pairwise注意力机制模型}
\subsection{基于递归注意力机制的模型}

\section{论文组织结构}

\begin{enumerate}
    \item 第一章介绍...
    \item 第二章介绍...
\end{enumerate}
    

\section{本章小结}


% \chapterbib

