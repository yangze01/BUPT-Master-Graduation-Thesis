%%
%% This is file `example/ch_intro.tex',
%% generated with the docstrip utility.
%%
%% The original source files were:
%%
%% install/buptgraduatethesis.dtx  (with options: `ch-intro')
%% 
%% This file is a part of the example of BUPTGraduateThesis.
%% 

\chapter{绪论}
\section{研究背景及意义}
随着互联网的普及和发展,各行各业都积累了大量的数据,为了应对信息爆炸带来的挑战,迫切需要一些自动化的手段帮助人们解决现实生活中的一些实际问题。如何通过计算机手段,将实际生活中人们的需求转换为计算机可解决的问题,成为我们需要越发关注的问题。

近年来,以大数据和人工智能为代表的信息科技快速发展,为法院信息化建设提供了有力的支撑。2016年7月,“智慧法院”建设被列入《国家信息化发展战略纲要》,作为国家深化电子政务,推进国家治理现代化、服务民主司法改革的重点任务。在“智慧法院”建设的大背景下,2016年11月,中国电子科技集团与最高人民法院联合成立“天平司法大数据有限公司”,充分运用司法大数据和人工智能技术,推进法院信息化、智能化建设。

传统的机器学习分类任务中,通常一个实例只包含一种标签,不同类别之间是互斥的,例如:人脸识别中,一个人的头像只能对应一种类别;新闻分类中,一篇新闻文本只对应于一种类型的新闻。而现实生活中的很多问题,往往一个实例对应多种标签,即多标签分类问题,例如:一个歌曲可以被划分为多个不同的流派;一张图片可能包含多种场景对象;生物信息学中,一个未标记的蛋白质含有多重功能;一部电影可能即属于战争又属于历史。在司法领域中,一个被告人可能触发了多个法条。这类针对一个实例拥有多个标签的任务被称为多标签分类任务。然而,在多标签分类研究中经常出现一些问题,比如不同的标签出现的次数可能会相差很多,标签之间存在相互之间的关联,分别称为标签不均衡问题和标签关联性问题。解决这些问题对多标签分类更好地应用于实际生活有重要意义。

司法判决预测可以根据事实描述进行判决结果预测,这项技术对司法辅助系统是十分有用的,一方面,它可以提供低成本,高质量的司法咨询服务;另一方面,它可以作为法官和律师等专业人士的参考。 因此,研究如何自动化地解决判决预测问题是十分重要的。




\section{国内外研究现状}
\subsection{司法辅助系统}

司法辅助系统作为司法领域不可或缺的一部分,为法官和普通民众提供了很多的帮助。目前已有很多智能系统应用于法律服务中。在国外,美国IBM公司基于IBM Watson设计了数字助理系统Ross,它通过理解自然语言,提供对案件的分析性的答案,已被超过10家主流律所雇佣。硅谷的Blackstone Discovery通过电子取证,以不超过10万美元的代价,在几天之内分析150万份法律文件。Beagle系统提供智能合同审核分析服务,用于削减风险,提高效率。英国伦敦的Hodge Jones设计了案件结果预测模型,用于评估人身伤害案件的胜诉可能性。直接导致2013年的Jackson民事诉讼改革,使诉讼成本大大降低。Lex Machina通过确定法官倾向性,基于对方律师过去处理的案件来形成相应的诉讼策略,并针对某个特定法院形成最有效的法律论证。英国的DoNotPay在线帮助用户挑战交通罚单,在纽约、伦敦和西雅图,已经成功挑战了超过20万个罚单,成功率达到60$\%$。目前已经涵盖到了航班延误补偿金请求、政府住房申请。在国内,小梨作为中国第一个机器人律师,提供签证、离婚等资讯服务。河北智审1.0系统开发于2016年,包括诉讼文书生成、案件信息回填、关联案件查询、大数据分析等功能。贵州执行阿尔法GO通过执行智库,资深法官智库,自动生成解决方案。北京最高人民法院2016年12月发布的睿法官,可以进行自动立案,庭审提纲生成,以及类案推送等功能。

作为司法智能化的传统任务,自动判决预测已经被研究了数十年。在早期的研究中,研究者通常将判决文书通过统计分析进行建模\cite{LiuC03, kort1957predicting, nagel1963applying, ulmer1963quantitative, keown1980mathematical},这些方法聚焦在如何用数学的方法对判决文书进行案例分析,而没有将判决预测考虑在内。Buchanan等人开创性地以人工智能的思想对法律为题进行定义,尝试从司法论证和推理角度对司法判决进行预测\cite{buchanan1970some}。现有的大多数现有的工作将这个任务当作文本分类问题进行处理,研究者通常通过从文本中抽取有效的特征并利用机器学习方法进行判决预测\cite{kim2014legal, AletrasTPL16, liu2015predicting}。Liu等人通过从历史数据中分抽取重要信息,采用KNN方法用于决定司法案例的诉讼理由\cite{LiuCH04}。Latz等人提出了一个基于树的模型用于预测美国最高法院法官的判决行为\cite{KatzBB14}。Sulea等人采用案例描述,规则以及案例时间作为特征信息,开发了一个采用多个SVM模型结果进行集成预测的系统\cite{Sulea2017Exploring}。Carvalho等人针对这个任务提出了一种两步方法,第一步,通过混合n-gram模型根据给定事实描述召回若干个法条,接下来,采用机器学习方法决定这些法条中哪些是真正相关的\cite{carvalho2015lexical}。这些方法需要花费大量的精力进行特征设计,并且很难进行大规模应用。

受到神经网络在各领域成功的启发\cite{Kim14, BordesGWB12, LuongSLVZ15},研究者开始将神经网络技术引入到判决预测任务中。Zhong等人将任务之间的依赖关系建模成一个有向图,并将这种拓扑结构引入判决预测任务中\cite{ZhongGTX0S18}。由于注意力机制在NLP任务中的成功应用,研究者开始将注意力机制引入模型用于处理判决预测,例如,Luo等人提出一种基于注意力机制的神经网络方法在一个统一的框架下联合建模判决预测任务和相关法条抽取任务\cite{luo2017learning}。Long等人利用注意力机制来建模事实描述,辩词,法条之间的复杂语义关系\cite{abs-1809-06537}。Hu等人将几种判别属性引入模型,增强事实描述和罪名之间的关系,提出一种属性-注意力机制罪名预测模型,用于同时预测法条和罪名\cite{C18-1041}。
% Wang等引入统一的动态成对注意力模型用于判决预测\cite{WangYNZZN18}。在他们的工作中,基于法条定义的成对注意力机制被引入分类模型,用于解决标签不均衡问题。

\subsection{多标签分类}
现有的多标签分类算法可以被划分为两个步骤:标签关联利用策略和阈值标定学习。第一个步骤主要用于获取标签之间的关联性,相关工作可以被划分为三类\cite{Zhang2014A}:一阶策略,二阶策略和高阶策略。例如Boutell等人将多标签分类问题划分为若干个独立的二分类问题\cite{Boutell2004Learning}。Brinker等人提出一种通用扩展方法,克服由于缺乏校准尺度引起的标签排序方法表达能力的限制\cite{Brinker2008Multilabel}。Tsoumakas等人提出了一种针对多标签分类的继承学习方法~\cite{Tsoumakas2007Random}。在他们的工作中,一种RAKEL算法被构造用于考虑标签集合中随机子集之间的影响。Li和Guo提出了一种利用核相关性分析捕获非线性标签相关性并为多标签学习执行非线性标签空间压缩的方法用于多标签分类\cite{Li2015Multi}。Zhai等人设计了一种以最小排序边界为目标函数的集成学习方法用来构造一个精准的多标签分类器\cite{Zhai2015A}。在第二个步骤,一个阈值学习方法被用来决定每个实例的标签集合大小。例如,Tsoumakas等人采用固定阈值的方法用来区分每个实例的相关和不相关标签\cite{Tsoumakas2007Random}。Yang~\cite{Yang2001}和Fan~\cite{fan2007study} 分析了在不同条件下几种阈值策略的性能。Elisseeff~\cite{Elisseeff2001A}和Zhang\cite{Zhang2014A}设计了一种线性回归模型用来预测标签集合的大小。

\subsection{注意力机制}
深度学习中的注意力机制模拟人脑在进行图像阅读和文本阅读机制,比如,当我们在看一副画的时候,我们的眼睛聚焦于图像的主体部分,而对背景等信息进行忽略;又比如,我们在阅读新闻的时候,会对新闻中的重点信息进行筛选,而忽略次要信息。这种机制首先在计算机视觉领域被使用~\cite{MnihHGK14}。

注意力机制在自然语言处理领域首先被机器翻译任务引入\cite{bahdanau2014neural},在他们的工作中,对源语言和目标语言使用注意力机制用于同时进行翻译和文本对齐。Luong等人扩展了之前的工作,提出了一种全局和局部注意力机制~\cite{LuongPM15}。这些方法都是基于递归神经网络的方法。随着注意力机制的在自然语言处理的广泛应用,研究者开始将注意力机制引入卷积神经网络,Yin等人在特征图上使用注意力机制以进行后续操作,并取得了良好的效果\cite{YinSXZ16}。

现有的很多研究工作中有很多基于自注意力机制的模型\cite{LinFSYXZB17, Jacob181004805},这种新型注意力机制模型通过自身的交互信息进行注意力加权。Vaswani等人使用自注意力机制构建了新的模型架构,提出了一种多头注意力机制\cite{VaswaniSPUJGKP17},得到了很好的效果。

\subsection{多任务学习}
同时学习多个任务的想法是通过利用在不同任务中所包含的不同信息来提高模型泛化性能。该方法广泛应用于各种领域,如计算机视觉\cite{Torralba2007Sharing, Yim2015Rotating, Zhang2013Robust},自然语言处理\cite{Liu2017Adversarial, Glorot2011Domain, collobert2008unified, Liu2015Representation, luong2015multi},基因工程\cite{Dong2014Inferring, Zhong2013User},表达学习\cite{argyriou2007multi, Kang2011Learning, Zhang2010Probabilistic}等。例如,张等人提出了一种多任务学习架构,它具有四种类型的递归神经层,可以跨多个相关任务融合信息~\cite{Zhang2017A}。Sun等人提出了一种面部识别和通用的联合模型,用于减少人际内部差异,同时扩大人际差异\cite{Sun2014Deep}。Wang等人引入了一个基于多任务学习的框架,用于学习子空间分割的耦合和非平衡表示\cite{Wang2015Multi}。提出了一个多任务学习的总体框架,使用递进学习进行句子提取和文档分类\cite{Masaru2017Masaru}。Misra等人的工作中介绍了一种使用多任务学习在ConvNets中学习共享表示的基础方法\cite{Misra2016Cross}。Pentina等人针对多任务学习中标注数据的可变性进行了研究\cite{Pentina2017Multi}。Collobert等人提出了一种统一的网络结果用于同时学习多种不同的任务\cite{collobert2008unified}。Li等人提出了一种对抗多任务学习框架来利用共享和私有隐含特征空间来帮助不同的任务提升性能\cite{Li2016Self}。

\section{研究内容}
本文的研究内容是面向司法领域多标签分类的研究与实现,主要解决在司法领域中挖掘事实描述与涉及的法条之间的关系,采用多标签分类的法条进行法条推荐,为法官裁决和一般民众的法律咨询提供智能化服务的基础。目前,国外已经有一部分针对司法领域犯罪行为与涉及法条之间关系智能化的研究,而国内是司法领域的人工智能研究刚刚开始,还有很高的研究价值。

标签不均衡问题和标签关联性问题是多标签分类中的两个常见问题,它们之间是相互促进又相互影响的,通过标签关联性可以一定程度上缓解标签不均衡问题,标签不均衡问题对标签关联性的获取又有一定程度的阻碍,因此,对这两方面的研究是十分重要的。本文从这两个基本问题出发,主要的研究内容和创新点如下:

\subsection{基于动态阈值的成对注意力机制模型}
司法领域的多标签分类主要面临两个方面的挑战,一个是针对给定相关的法条数量的动态的,我们定义为标签动态问题。另一个是大多数标签很少被命中,我们称为标签不均衡问题。之前的工作通常独立地学习多标签分类模型和标签阈值,并且忽略了标签不均衡问题。为了解决这两个挑战,本文提出了一种通用成对注意力机制模型(简称\textbf{DPAM})。具体地,\textbf{DPAM}采用多任务学习框架联合学习多标签分类器和阈值预测器,因此\textbf{DPAM}可以通过学习两个任务的交互信息以提升模型的泛化性能。此外,通过引入基于法条定义的成对注意力模型用来缓解标签不均衡问题。

\subsection{基于递归注意力机制的模型}
法官在进行案件审判的过程中,经常需要来回重复阅读事实描述和法条定义来找到有效信息以便进行正确的匹配(例如,针对给定事实描述,决定与之相关的法条),现有的司法领域多标签分类算法在进行法条推荐的过程中,仅仅分析了相关法条定义的表层语义信息,往往忽略了事实描述和法条之间的重复的交互信息。本文针对这个问题,提出了一种基于递归注意力机制的模型(简称\textbf{RAN}),\textbf{RAN}通过模拟法官交替阅读事实描述和法条定义的过程,引入两者之间的潜在语义信息,从而提升模型性能。
\section{论文组织结构}
本文的组织结构和章节安排如下:

第一章绪论。首先介绍了面向司法领域多标签分类问题的研究背景及研究意义,之后阐述当前国内外司法辅助系统以及与本课题相关的一些内容的国内外研究现状,接着介绍本文工作的主要研究内容和创新点,最后给出本文的具体章节安排。

第二章相关概念与相关技术。本章节主要介绍本文用到的相关技术以及一些基本概念。首先针对本文研究的问题,详细介绍本文研究问题的定义以及评价指标。确定本文研究问题的输入及输出之后,从三类方法介绍文本分类的基础(文本表示),最后给出本文用到的两种流行技术,注意力机制和多任务学习技术。

第三章是基于动态阈值的成对注意力机制模型。首先介绍算法的动机。之后详细介绍算法的设计,实验数据集,实验结果对比以及实验结果分析。

第四章是基于递归注意力机制的模型。首先介绍算法的实现动机。之后详细阐述算法的设计,实验数据集,实验结果对比以及实验结果分析。

第五章是原型系统的设计与实现。本章基于提出的模型,设计实现了一个用于验证功能的原型系统。首先介绍了系统的整体功能设计,之后对涉及本课题模型的部分进行了功能测试。

第六章是总结与展望。这一章针对本文的内容进行总结,回顾研究内容和研究成果,并对未来工作进行展望。


\section{本章小结}
本章首先介绍了面向司法领域多标签分类问题的研究背景,国内外研究现状。之后根据现有研究的不足,提出了本文的研究内容。为了解决现有算法中存在的问题以及模型的不足,本文提出了一种基于动态阈值的成对注意力机制模型和一种基于递归注意力机制的模型。最后描述了本课题整体章节结构。

% \chapterbib

